\documentclass[letterpaper]{ltxdoc}

% Set Latin Modern as the main font family
\usepackage{lmodern}
% Enable UTF-8 input encoding
\usepackage[T1]{fontenc}
% Enable proper font encoding (e.g., accents)
\usepackage[utf8]{inputenc}
% Enable hyperlinks in the document
\usepackage{hyperref}
% Enable syntax highlighting source code
\usepackage{minted}

% ===========================
% Settings
% ===========================

% Settings for the 'hyperref' package
\hypersetup{
    pdftitle={A classe uecetex2},
    pdfauthor={Thiago Ferreira},
    pdfsubject={Documentos técnicos e científicos brasileiros para universidade estadual do ceará},
    pdfkeywords={uece}{uecetex2}{abntex2},
    pdfproducer={Thiago Ferreira},
    pdfcreator={LaTeX with ueceTeX2},
    colorlinks=true,
    linkcolor=blue,
    citecolor=blue,
    urlcolor=blue
}

% Settings for the 'minted' package
\setminted{
    linenos,
    frame=single,
    fontsize=\footnotesize,
}

% ===========================
% Document Information
% ===========================

\title{The \textsf{uecetex2} Class}

\author{Thiago Ferreira}

\date{<CURRENT_DATE>, <VERSION>}

% ===========================
% Main Document
% ===========================

\begin{document}

\maketitle

\begin{abstract}
    This is an abstract teste
\end{abstract}

\tableofcontents

\section{Considerações iniciais}

\DescribeMacro{\documentclass}
A classe \textsf{abntex2} foi criada como um conjunto de configurações da classe
\textsf{memoir}\footnote{A versão anterior do ~era baseada na classe
\textsf{report}.} \cite{memoir}. Desse modo, todas as opções do \textsf{memoir}
estão disponíveis, como por exemplo, |12pt,openright,twoside,a4paper,article|.
Consulte o manual do \textsf{memoir} para outras opções.

As opções mais comuns de inicialização do texto do documento são:

\begin{minted}{latex}
\documentclass[]{uecetex2}
\end{minted}

\end{document}
